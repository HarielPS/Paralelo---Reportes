\chapter[Soluciones individuales de series de Fourier]{Soluciones individuales de series de Fourier}
\label{cp:user-guide}

{
\parindent0pt

En este apartado se presentan las soluciones individuales de cada estudiante, desarrolladas a partir de las funciones propuestas por el profesor de la materia de Cómputo Paralelo.El objetivo principal de esta fase fue llevar a cabo el cálculo de la serie de Fourier correspondiente a cada función asignada, aplicando los métodos matemáticos adecuados para la obtención de los coeficientes y la expansión de la función en términos de senos y cosenos.
\vspace{10pt}

Cada integrante del equipo realizó los cálculos de manera independiente, asegurando un desarrollo detallado y estructurado de los coeficientes de Fourier. Para ello, se siguió el procedimiento estándar de integración para la obtención de los coeficientes \(a_0\), \(a_n\) y \(b_n\), que determinan la representación trigonométrica de la función en términos de series de Fourier.
\newpage
% \begin{verbatim}
% 00-Abstract.tex
% 01-Introduction.tex
% 02-User-Guide.tex
% ...
% \end{verbatim}
%---------------------------------------- Ilse

\section{Función resuelta por Castro Paez Ilse Yazbeth}
En este siguiente apartado, se presentan los cálculos para de la serie de Fourier de la función \(f(x)=9-3x-x^2\), correrpondientes a la figura \ref{fig:figure-ilse-01}, donde se muestran detalladamente el desarrollo hasta obtener el resultado. 

\begin{figure}[H]
    \centering
    \includegraphics[width=\linewidth]{Figures/fourierIlse/fourierIlse.jpg}
    \caption[Cálculo de la función \(f(x)=9-3x-x^2\)]{Cálculo de la función \(f(x)=9-3x-x^2\), resuelto por Castro Paez Ilse Yazbeth}
    \label{fig:figure-ilse-01}
\end{figure}



%---------------------------------------- Daniel
\newpage
\section{Función resuelta por Catonga Tecla Daniel Isaí}
En el siguiente apartado, se presentan los cálculos para la obtención de los coeficientes de Fourier de la función asignada, \(f(x)=6-2x\), correspondientes a las figuras \ref{fig:figure-daniel-01} hasta \ref{fig:figure-daniel-04}. Se detallan las ecuaciones utilizadas y el procedimiento matemático seguido para su determinación.

\begin{figure}[H]
    \centering
    \includegraphics[width=\linewidth]{Figures/fourierDaniel/fourierDaniel1.jpg}
    \caption[Cálculo de \(a_0\) para \(f(x)=6-2x\)]{Cálculo de \(a_0\) para \(f(x)=6-2x\), resuelto por Catonga Tecla Daniel Isaí}
    \label{fig:figure-daniel-01}
\end{figure}

\begin{figure}[H]
    \centering
    \includegraphics[width=\linewidth]{Figures/fourierDaniel/fourierDaniel2.jpg}
    \caption[Cálculo parcial de \(a_n\) para \(f(x) = 6 - 2x\)]{Desarrollo parcial del cálculo de los coeficientes \(a_n\) para la función \(f(x) = 6 - 2x\), resuelto por Catonga Tecla Daniel Isaí. El resultado final aún no se muestra.}
    \label{fig:figure-daniel-02}
\end{figure}

\begin{figure}[H]
    \centering
    \includegraphics[width=\linewidth]{Figures/fourierDaniel/fourierDaniel3.jpg}
    \caption[Cálculo de \(a_n\) y desarrollo de \(b_n\) para \(f(x) = 6 - 2x\)]{Cálculo de \(a_n\) y desarrollo parcial de \(b_n\) para la función \(f(x) = 6 - 2x\), resuelto por Catonga Tecla Daniel Isaí.}
    \label{fig:figure-daniel-03}
\end{figure}

\begin{figure}[H]
    \centering
    \includegraphics[width=\linewidth]{Figures/fourierDaniel/fourierDaniel4.jpg}
    \caption[Cálculo de \(b_n\) para \(f(x) = 6 - 2x\)]{Cálculo de \(b_n\) para la función \(f(x) = 6 - 2x\), mostrando la función reconstruida a partir de los coeficientes, resuelto por Catonga Tecla Daniel Isaí.}
    \label{fig:figure-daniel-04}
\end{figure}


%-------------------------------------------- Hariel
\newpage
\section{Función resuelta por Padilla Sanchez Hariel }
     En esta sección, se expone el procedimiento matemático para determinar los coeficientes de Fourier de la función \(f(x)=6-4x\). Se inicia con la ecuación general de la serie de Fourier y se procede al cálculo del coeficiente \(a_0\) como se muestra en la figura \ref{fig:figure-hariel-01},Para la obtención de \(a_n\), se emplea integración por partes, evidenciando la cancelación de ciertos términos. Se incluyen además observaciones sobre las propiedades trigonométricas utilizadas en el análisis.

%figura de a0 y an de Hariel 
    \begin{figure}[H]
        \centering
        \includegraphics[width=\linewidth]{Figures/fourierHariel/fase1/funcion 1.jpg}
        \caption[Cálculo de \(a_0\) y \(a_n\) para \(f(x) = 6 - 4x\)]{Cálculo de \(a_0\) y \(a_n\) para la función \(f(x) = 6 - 4x\), resuelto por Padilla Sánchez Hariel.}
        \label{fig:figure-hariel-01}
    \end{figure}

    En el siguiente apartado, se presentan los cálculos para la obtención del coeficiente \(b_n\) de la serie de Fourier de la función \(f(x)=6-4x\). Se muestra el desarrollo de la integral correspondiente y la aplicación del método de integración por partes para resolver términos específicos como se muestra en la figura \ref{fig:figure-hariel-02}. Finalmente, se obtiene una expresión general para \(b_n\), la cual se utilizará en la reconstrucción de la función periódica.

%Figura de bn de Hariel
    \begin{figure}[H]
        \centering
        \includegraphics[width=\linewidth]{Figures/fourierHariel/fase1/funcion 2.jpg}
        \caption[Cálculo de \(b_n\) para \(f(x) = 6 - 4x\)]{Cálculo de \(b_n\) para la función \(f(x) = 6 - 4x\), resuelto por Padilla Sánchez Hariel.}
        \label{fig:figure-hariel-02}
    \end{figure}

%-------------------------------------------- Manuel
\section{Función resuelta por Olguin Castillo Víctor Manuel }
    A continuacion se calculan los coeficientes de Fourier asociados a la función \(f(x)=x^2-3x-3\) . 

%figura de a0
    \begin{figure}[H]
        \centering
        \includegraphics[width=\linewidth]{Figures/fourierManuel/a0.jpeg}
        \caption[Cálculo de \(a_0\)]{Cálculo de \(a_0\), resuelto por Olguin Castillo.}
        \label{fig:figure-manuel-01}
    \end{figure}

%figura de a0
    \begin{figure}[H]
        \centering
        \includegraphics[width=\linewidth]{Figures/fourierManuel/an.jpeg}
        \caption[Cálculo de \(a_n\)]{Cálculo de \(a_n\), resuelto por Olguin Castillo.}
        \label{fig:figure-manuel-02}
    \end{figure}

\begin{figure}[H]
        \centering
        \includegraphics[width=\linewidth]{Figures/fourierManuel/bn.jpeg}
        \caption[Cálculo de \(b_n\)]{Cálculo de \(b_n\), resuelto por Olguin Castillo.}
        \label{fig:figure-manuel-03}
    \end{figure}

El método se basa en la expansión en series de Fourie usando simplificaciones algebraicas y las propiedades trigonométricas que permiten reducir la expresión final.



%-------------------------------------------- Fase 3
\section{Programa en C para el cálculo de la serie de Fourier}
    El objetivo del programa es calcular una aproximación de la serie de Fourier para la función \(f(x)=6−4x\) ., utilizando un enfoque paralelo basado en la creación de procesos hijos, memoria compartida y semáforos en el sistema operativo Linux. Esta función es impar, por lo tanto, su serie de Fourier se puede expresar únicamente con términos en seno, en la forma:
\[
f(x) \approx a_0 + \sum_{n=1}^{N} b_n \sin(nx)
\]

donde \( a_0 = 6 \) y los coeficientes \( b_n \) están dados por:

\[
b_n = \frac{8}{n}(-1)^n
\]

Para este programa, se consideran valores de \( x \) desde -3.14 hasta 3.14 con un incremento de 0.15. Se calculan los primeros 10 términos de la serie de Fourier, y cada uno de estos es calculado por un proceso hijo independiente. Los resultados son almacenados en una matriz en memoria compartida, protegida por un semáforo para evitar conflictos en la escritura concurrente. Finalmente, se suman todos los términos y se genera un archivo CSV con los resultados.


Para la implementación en C se hace uso de varias bibliotecas. La biblioteca \texttt{stdio.h} se emplea para las operaciones de entrada y salida estándar como la lectura y escritura en archivos. La biblioteca \texttt{stdlib.h} se utiliza para funciones de control como \texttt{exit()} y la gestión de memoria dinámica. \texttt{unistd.h} proporciona el uso de \texttt{fork()}, fundamental para la creación de procesos hijos. Además, se incluyen \texttt{sys/ipc.h}, \texttt{sys/shm.h} y \texttt{sys/sem.h}, necesarias para trabajar con memoria compartida y semáforos bajo el esquema de IPC System V en Linux. Finalmente, \texttt{math.h} permite el uso de funciones matemáticas como \texttt{sin()} y \texttt{pow()}.

\begin{figure}[H]
    \centering
    \includegraphics[width=0.8\linewidth]{Figures/codigo/librerias.png}
    \caption[Librerías utilizadas en el programa]{Fragmento de código que muestra las librerías utilizadas en el programa.}
    \label{fig:codigo-funcion}
\end{figure}


La función \texttt{funcion} tiene la responsabilidad de calcular cada término individual de la serie de Fourier. Recibe como parámetros el valor de \( n \) y el punto \( x \), y devuelve el resultado de la expresión \( \frac{8}{n} (-1)^n \sin(nx) \), que corresponde al coeficiente \( b_n \sin(nx) \). Este valor es calculado por cada proceso hijo para los diferentes valores de \( x \).

\begin{figure}[H]
    \centering
    \includegraphics[width=0.8\linewidth]{Figures/codigo/funcion.png}
    \caption[Función para calcular \(b_n \sin(nx)\)]{Fragmento de código de la función que calcula el término \(b_n \sin(nx)\).}
    \label{fig:codigo-funcion}
\end{figure}

Para controlar el acceso concurrente a la memoria compartida, se definen dos funciones auxiliares: \texttt{down} y \texttt{up}. La función \texttt{down} realiza una operación de espera o bloqueo, decrementando el valor del semáforo. Esto indica que un proceso ha ingresado a la sección crítica. Por otro lado, la función \texttt{up} incrementa el semáforo, liberando así la sección crítica para que otro proceso pueda acceder. Estas funciones garantizan la exclusión mutua al momento de escribir en la memoria compartida.

La función \texttt{Crea\_semaforo} encapsula el proceso de creación de un semáforo. Se basa en una clave generada por la función \texttt{ftok()} y lo crea con permisos de lectura y escritura para todos los usuarios. Se inicializa con un valor de 1, lo cual significa que inicialmente la sección crítica está libre.

\begin{figure}[H]
    \centering
    \includegraphics[width=0.9\linewidth]{Figures/codigo/semaforo.png}
    \caption[Creación del semáforo]{Fragmento de código que muestra la creación del semáforo.}
    \label{fig:semaforo}
\end{figure}

La función \texttt{crearArchivo} se utiliza para crear archivos vacíos que posteriormente son usados por \texttt{ftok()} para generar claves únicas. Aunque no escriben información dentro del archivo, su existencia es necesaria para que \texttt{ftok()} funcione correctamente. Se crean archivos para la memoria compartida y el semáforo.

\begin{figure}[H]
    \centering
    \includegraphics[width=0.9\linewidth]{Figures/codigo/crearArchivo.png}
    \caption[Creación de archivos vacíos]{Fragmento de código que muestra la creación de archivos vacíos para la generación de claves.}
    \label{fig:archivo_crear}
\end{figure}

Dentro de la función \texttt{main}, se calcula primero la cantidad de puntos de evaluación entre los valores de -3.14 y 3.14, considerando un incremento de 0.15. Esta cantidad determina el tamaño del arreglo de valores de \( x \), así como el tamaño de la memoria compartida necesaria para guardar todos los resultados. Luego, se generan las claves para la memoria y el semáforo, y se crean ambos recursos. A continuación, se calcula el arreglo de valores de \( x \) y se almacena en un arreglo auxiliar.

\begin{figure}[H]
    \centering
    \includegraphics[width=0.9\linewidth]{Figures/codigo/x_values.png}
    \caption[Generación de valores de \(x\)]{Cálculo de los valores de \(x\) que se usarán en la evaluación de los términos de la serie.}
    \label{fig:x-values}
\end{figure}

Posteriormente, se crea el primer proceso hijo que escribe directamente la constante \( a_0 = 6 \) en toda la primera fila de la matriz compartida. Después, se crean otros diez procesos hijos. Cada uno de estos se encarga de calcular los valores del término \( b_n \sin(nx) \) para cada valor de \( x \). Para evitar que múltiples procesos escriban al mismo tiempo, cada proceso llama a \texttt{down} antes de escribir y a \texttt{up} después de terminar de escribir.

\begin{figure}[H]
    \centering
    \includegraphics[width=0.9\linewidth]{Figures/codigo/procesos_hijos.png}
    \caption[Creación y sincronización de procesos]{Código que muestra la creación de procesos hijos y el uso de semáforo para escritura sincronizada.}
    \label{fig:procesos-hijos}
\end{figure}


El proceso padre se encarga de esperar a que todos los procesos hijos terminen su ejecución. Para ello, utiliza múltiples llamadas a la función \texttt{wait()}. Una vez que todos los hijos han terminado, el padre realiza la suma de todos los términos calculados para cada valor de \( x \). Esta suma se almacena en la última fila de la matriz compartida.

\begin{figure}[H]
    \centering
    \includegraphics[width=0.9\linewidth]{Figures/codigo/suma_resultados.png}
    \caption[Suma de los resultados y generación de archivo]{Código que suma los términos de la serie para cada valor de \( x \) y escribe los resultados en un archivo.}
    \label{fig:suma-resultados}
\end{figure}

Finalmente, se abre un archivo CSV llamado \texttt{resultados.csv} y se imprimen en él todos los resultados, incluyendo el valor de \( x \), cada término de la serie y la suma total.

\begin{figure}[H]
    \centering
    \includegraphics[width=0.9\linewidth]{Figures/codigo/escritura_archivo.png}
    \caption[Escritura de resultados en archivo CSV]{Código que muestra la escritura de los resultados en un archivo CSV.}
    \label{fig:escritura-archivo}
\end{figure}


Al finalizar la ejecución del programa, se imprime en la consola la matriz completa de resultados. En cada línea se muestra un valor de \(x\), seguido por los términos \(a_0\) y \(b_n \sin(nx)\) calculados por cada proceso hijo para ese punto, así como la suma total de todos los términos, que representa la aproximación de \(f(x)\) mediante la serie de Fourier. Esta visualización permite al usuario verificar que los cálculos se han realizado correctamente y apreciar el comportamiento de la aproximación en el dominio especificado.

\begin{figure}[H]
    \centering
    \includegraphics[width=0.9\linewidth]{Figures/codigo/impresion_matriz.png}
    \caption[Impresión de la matriz de resultados]{Código que muestra la impresión de la matriz de resultados en la consola.}
    \label{fig:impresion-matriz}
\end{figure}

Como parte del cierre correcto del programa, se realiza la desvinculación de la memoria compartida mediante \texttt{shmdt()}, y posteriormente se elimina del sistema con \texttt{shmctl()}. Asimismo, se elimina el semáforo con \texttt{semctl()}, liberando así todos los recursos del sistema utilizados por el programa.

\begin{figure}[H]
    \centering
    \includegraphics[width=0.9\linewidth]{Figures/codigo/cierre.png}
    \caption[Cierre del programa y liberación de recursos]{Código que muestra el cierre del programa y la liberación de recursos.}
    \label{fig:cierre-programa}
\end{figure}


\section{Análisis de la aproximación mediante serie de Fourier}

Se presenta un análisis sobre la aproximación mediante series de Fourier para la función lineal:

\[f(x) = 6 - 4x\]


El analisis se apoya en los calculos de la serie de fourier realizados por el estudiante Padilla Sanchez Hariel \ref{fig:figure-hariel-01} y \ref{fig:figure-hariel-02}. La función \(f(x)\) es lineal y se define en el intervalo \([-3.14, 3.14]\). La serie de Fourier para esta función se expresa como:

\[\text{Término}_n = \frac{8}{n} \cdot (-1)^n \cdot \sin(n \cdot x)\]

Una vez calculados los coeficientes \(a_0\) y \(b_n\), se procede a la evaluación de la serie en un rango de valores de \(x\). De primera mano se realizaron calculos en el intervalos sobre la ecuacion \[f(x) = 6 - 4x\] esto nos dara una recta lineal, posterior a ello durante la fase 2 se realizaron los calculos de la serie de fourier en una hoja de excel, donde se implementaron las fórmulas necesarias para calcular los términos de la serie y finalmente se realizaron los calculos con el programa en C usando memoria compartida y semáforos para la sincronización de procesos como se muestra en \ref{fig:impresion-matriz}.

Se juntaron los resultados obtenidos de la hoja de excel y del programa en C, y se graficaron en la figura \ref{fig:grafica-fase3}. En esta gráfica se observa que la ecuacio nos muestra una funcion lineal en la recta naranja, mientras que las series de fourier calculadas en el intervalo tanto como en el excel como se muestra pintada en azul, y el programa en C pintada en verde, nos muestran una aproximacion a la funcion lineal, sin embargo, se puede observar que la serie de fourier tiene mucha similitud a las funcion sin embargo en la parte de los extremos se puede observar que la serie de fourier no logra aproximarse a la funcion lineal, esto es debido a que la serie de fourier es una aproximacion de la funcion en un intervalo cerrado, y al ser una funcion lineal no tiene un periodo definido, por lo que la serie de fourier no logra aproximarse a la funcion en su totalidad.

\begin{figure}[H]
    \centering
    \includegraphics[width=0.9\linewidth]{Figures/fase3/grafica_fase3.png}
    \caption[Grafica de resultados sobre los calculos en excel y el codigo C.]{Grafica de resultados sobre los calculos en excel y el codigo C.}
    \label{fig:grafica-fase3}
\end{figure}

%-------------------------------------------- Fase 4
\section{Programa en C para el cálculo de la serie de Fourier usando hilos}

En esta fase se desarrolló un programa en lenguaje C que implementa el cálculo de la serie de Fourier para la función $f(x) = 6 - 4x$ utilizando programación paralela con hilos mediante la biblioteca \texttt{pthread}. A diferencia de las fases anteriores en las que se usaron procesos hijos y mecanismos de sincronización como semáforos, los hilos permiten una forma más sencilla de compartir memoria y comunicación entre tareas paralelas, ya que todos los hilos comparten el mismo espacio de direcciones dentro del proceso principal. Esto reduce la complejidad y el uso de estructuras adicionales para el intercambio de información.

El programa inicia con la inclusión de bibliotecas esenciales para su funcionamiento. Se utiliza \texttt{stdio.h} para las operaciones de entrada y salida como impresión por pantalla y escritura en archivos. \texttt{stdlib.h} se usa para funciones relacionadas con la administración de memoria dinámica. \texttt{math.h} se emplea para funciones matemáticas como \texttt{sin()} y \texttt{pow()} que son necesarias para calcular los términos de la serie. Finalmente, \texttt{pthread.h} es crucial para manejar la creación, ejecución y sincronización de los hilos. La inclusión de estas bibliotecas asegura que el programa pueda realizar correctamente todas las operaciones necesarias para el cálculo paralelo de la serie de Fourier.

Las constantes definidas al inicio del programa delimitan el comportamiento del mismo. Se declara \texttt{INICIO} con un valor de 3.14 y \texttt{FIN} con -3.14, indicando que el intervalo de evaluación de la función se realiza simétricamente alrededor del cero. El paso, definido por la constante \texttt{PASO}, se establece en 0.15, determinando la resolución del muestreo de $x$. La constante \texttt{COLUMNAS} se define como 12, pues considera una fila para el valor $a_0$, diez filas para los términos $b_n \sin(nx)$ desde $n = 1$ hasta $n = 10$, y una fila adicional para almacenar la suma total de todos los términos, que representa el valor aproximado de $f(x)$ en cada punto.

\begin{figure}[H]
    \centering
    \includegraphics[width=0.9\linewidth]{Figures/fase4/librerias_fase4.png}
    \caption{Fragmento de código donde se incluyen las bibliotecas y se definen las constantes utilizadas.}
    \label{fig:librerias-fase4}
\end{figure}

La función principal del programa tiene como responsabilidad inicial determinar cuántos puntos se evaluarán en el intervalo definido. Esto se logra calculando el número de pasos de $x$ a través de la fórmula: 
\[
\texttt{filas} = \frac{|\texttt{INICIO} - \texttt{FIN}|}{\texttt{PASO}} + 1
\]
El resultado determina la cantidad de columnas de la matriz bidimensional donde se almacenarán los resultados, ya que cada columna representará un valor diferente de $x$.

Se define una estructura llamada \texttt{hilo\_args\_t} que encapsula la información que cada hilo necesita para realizar su cálculo de forma independiente. Esta estructura contiene un puntero a la matriz compartida de resultados, el índice del término que le corresponde al hilo (por ejemplo, si se trata de $a_0$, $b_1\sin(x)$, $b_2\sin(2x)$, etc.), el número total de filas y el valor del paso. Gracias a los hilos compartir memoria con el hilo principal, no se requiere el uso de memoria compartida explícita, lo que simplifica el diseño del programa.

\begin{figure}[H]
    \centering
    \includegraphics[width=0.9\linewidth]{Figures/fase4/estructura_fase4.png}
    \caption{Definición de la estructura utilizada para pasar parámetros a cada hilo.}
    \label{fig:estructura-fase4}
\end{figure}

Cada hilo ejecuta la función \texttt{calcular\_fila}, que es responsable de llenar una fila específica de la matriz. Si el índice recibido es 0, el hilo escribe el valor constante $a_0 = 6$ en toda la fila. En caso contrario, se calcula el valor de $b_n$ según la fórmula $b_n = \frac{8}{n}(-1)^n$ y se evalúa $b_n \sin(nx)$ para cada valor de $x$ en el intervalo definido. La variable $x$ se actualiza a partir del índice de la fila y el valor de paso, es decir, $x = \texttt{INICIO} - i \cdot \texttt{PASO}$. Este diseño permite que cada hilo trabaje de forma independiente sobre una porción distinta del resultado final.

\begin{figure}[H]
    \centering
    \includegraphics[width=0.9\linewidth]{Figures/fase4/calcular_fila_fase4.png}
    \caption{Función que calcula cada fila de la matriz con base en el índice del hilo.}
    \label{fig:calcular-fila-fase4}
\end{figure}

Una vez definida la función que ejecuta cada hilo, en la función principal se crean 12 hilos correspondientes a cada fila de la matriz. A cada hilo se le asigna una instancia de la estructura de parámetros, se inicializa con los valores necesarios y se invoca usando \texttt{pthread\_create}. Posteriormente, se usa \texttt{pthread\_join} para esperar la finalización de cada hilo antes de continuar, asegurando que todos los resultados estén correctamente calculados antes de proceder al siguiente paso.

\begin{figure}[H]
    \centering
    \includegraphics[width=0.9\linewidth]{Figures/fase4/hilos_fase4.png}
    \caption{Fragmento del código que muestra la creación y sincronización de los hilos.}
    \label{fig:hilos-fase4}
\end{figure}

Después de que todos los hilos han terminado su ejecución, se calcula la suma total por columnas, almacenando el resultado en la última fila de la matriz. Esta fila representa la aproximación de la función $f(x)$ mediante la suma de los 11 términos de la serie de Fourier evaluados para cada valor de $x$. La matriz se recorre iterativamente sumando el valor correspondiente de cada fila (excepto la última) para cada columna, y el resultado se guarda en la posición correspondiente de la fila final.

\begin{figure}[H]
    \centering
    \includegraphics[width=0.9\linewidth]{Figures/fase4/suma_total_fase4.png}
    \caption{Código encargado de calcular la suma total de la serie de Fourier en cada punto.}
    \label{fig:suma-total-fase4}
\end{figure}

Una vez completados todos los cálculos, los resultados se escriben en un archivo CSV llamado \texttt{resultado\_hilos.csv}. Cada fila del archivo corresponde a un término de la serie (incluyendo la suma total), y cada columna representa un valor de $x$. Este archivo puede ser utilizado posteriormente para generar gráficas comparativas o para su análisis en herramientas como Excel o Python.

\begin{figure}[H]
    \centering
    \includegraphics[width=0.9\linewidth]{Figures/fase4/guardar_csv_fase4.png}
    \caption{Fragmento de código encargado de guardar los resultados en un archivo CSV.}
    \label{fig:guardar-csv-fase4}
\end{figure}

Además del archivo CSV, el programa también imprime los resultados en la terminal, mostrando para cada valor de $x$ el valor aproximado de $f(x)$ obtenido mediante la suma de la serie de Fourier. Esta salida permite verificar rápidamente si los resultados parecen razonables sin necesidad de abrir el archivo de salida.

\begin{figure}[H]
    \centering
    \includegraphics[width=0.9\linewidth]{Figures/fase4/mostrar_terminal_fase4.png}
    \caption{Impresión de los resultados aproximados de $f(x)$ en la terminal.}
    \label{fig:mostrar-terminal-fase4}
\end{figure}

Finalmente, se observa una gráfica comparativa entre las tres fases implementadas: la primera con una función lineal, la segunda con los valores de excel, la tercera con procesos y la cuarta con hilos. En esta gráfica se representa el valor original de la función $f(x) = 6 - 4x$ y las aproximaciones obtenidas en cada fase. Se observa que, tanto los resultado generados en la fase 2 calculados por medio de excel, los valores de la fase 3 con procesos y los de la fase 4 con hilos, todos son iguales y asu vez se aproximan a la función original.

\begin{figure}[H]
    \centering
    \includegraphics[width=0.9\linewidth]{Figures/fase4/grafica_comparativa_fase4.png}
    \caption{Comparativa gráfica entre el valor real de la función y las aproximaciones de las fases 2, 3 y 4.}
    \label{fig:grafica-comparativa-fase4}
\end{figure}
