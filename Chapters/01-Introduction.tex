\chapter[Introducción]{Introducción}
\label{cp:introduction}

{
\parindent0pt

Uno de los aspectos clave del análisis de Fourier es la obtención de los coeficientes que permiten reconstruir una función a partir de sus componentes armónicas. Estos coeficientes determinan la contribución de cada frecuencia en la representación de la función original y son esenciales para aplicaciones como el procesamiento de señales, la resolución de ecuaciones diferenciales y la compresión de datos.
\vspace{10pt}

En este proyecto, se estudia el cálculo de los coeficientes de Fourier para una función específica, analizando el procedimiento matemático necesario para su determinación. A través de este desarrollo, se busca comprender cómo el análisis de Fourier permite representar funciones de manera eficiente y cómo esta técnica se aplica en diferentes áreas del conocimiento.
\vspace{10pt}

Se incluye un marco teórico que explora las aplicaciones más relevantes del análisis de Fourier, destacando su impacto en la ciencia y la tecnología.
\vspace{10pt}

Este trabajo tiene como objetivo reforzar la comprensión de las series de Fourier y su utilidad en el estudio de fenómenos periódicos, proporcionando una base sólida para su aplicación en problemas reales. A través de la implementación práctica del cálculo de coeficientes de Fourier, se demuestra cómo esta herramienta facilita el análisis de funciones y contribuye a desarrollar soluciones más eficientes en diversas disciplinas científicas e ingenieriles.
\vspace{10pt}

El análisis de Fourier y el cálculo de sus coeficientes son herramientas fundamentales en las matemáticas, con un impacto significativo en diversas disciplinas, que abarcan desde la ingeniería hasta las ciencias sociales. Esta proyecto tiene como objetivo no solo desarrollar la teoría subyacente a este proceso, sino también profundizar en la importancia de aplicar las series de Fourier en la resolución de problemas complejos. A través de este enfoque, se pretende demostrar cómo estas series permiten transformar funciones en distintas áreas del conocimiento, facilitando su análisis y solución.




}