\chapter[Conclusiones]{Conclusiones}
\label{cp:latex-tutorial}

{
\setlength{\parskip}{1\baselineskip}
\parindent0pt

\newpage

\section{Castro Paez Ilse Yazbeth}

Resolver la serie de Fourier fue un proceso laborioso pero muy interesante. Al tratarse de una función polinómica, al inicio parecía ser muy sencilla pero conforme avancé en los cálculos  me percate de la importancia de aplicar correctamente cada uno de los pasos del desarrollo de Fourier. 

Desde el planteamiento de la integral hasta la obtención de los coeficientes, cada una de las operaciones requería de precisión y un buen entendimiento sólido de cómo usar correctamente el método. En particular, resolver la integral de una función cuadrática presentó un reto interesante, ya que implicó manipular expresiones algebraicas de segundo grado y así verificar que los resultados fueran consistentes. Otro punto importante fue saber interpretar correctamente el resultado de la serie y cómo la combinación de términos trigonométricos podía aproximar la función original de forma periódica. A parte de los cálculos de la función, esta actividad me ayudó y me resultó bastante útil para poder comprender de mejor manera el contexto y la utilidad de las series de Fourier. 

En el área de la inteligencia artificial juega un papel de gran importancia en el procesamiento de señales y de imágenes, especialmente en técnicas de filtrado y compresión de datos. Dentro del cómputo paralelo, su optimizada implementación permite que el proceso del análisis de grandes volúmenes de información sea más veloz, lo que facilita las aplicaciones en las telecomunicaciones y el procesamiento del audio.

Este ejercicio me dejó una mayor apreciación y un mayor aprendizaje en la profundidad matemática detrás del análisis de Fourier. Aunque al inició el hecho de hacer una transformación de una función en una serie infinita de términos trigonométricos puede resultar ser algo abstracta, al ir desarrollando los cálculos comprendí mucho mejor cómo es que funciona esta descomposición y su importancia en diversos campos. 

Me permitió reforzar mis habilidades de integración y manipulación algebraica, lo cual es esencial para el correcto manejo de ecuaciones en aplicaciones más avanzadas. 
La manipulación igualmente desempeñó un papel de gran importancia en el proceso, desde la simplificación de expresiones hasta la ordenación de términos en las ecuaciones resultantes. Otra de las habilidades reforzadas fue la interpretación de los resultados obtenidos, ya que no basta solo con obtener una solución matemática, si no que es de gran importancia saber comprender su significado y sus aplicaciones. 

Este ejercicio me permitió conectar y entender mejor los conceptos abstractos con aplicaciones prácticas al mundo real. La resolución de la serie de Fourier para esta función representó un reto matemático interesante y también reforzó mis habilidades fundamentales en la integración, álgebra y el modelado matemático.

Como parte de la segunda fase, se realizó la graficación de la función proporcioanda usando series de fourier y una matriz en excel. Para ello se generaron los datos correspondientes a los coeficientes de la serie. Usando fórmulas y herramientas gráficas de Excel, se pudo visualizar la aproximación de la serie a la función original. 

Esta actividad reforzó mi comprensión teórica de las series de Fourier y me permitió observar cómo, al ir incrementndo el número de términos de la serie, la aproximación se vuelve cada vez mas precisa. 

Uno de los aspectos más destacados de esta experiencia fue el aprendizaje y la aplicación de fórmulas y funciones mas avanzadas en una hoja de cálculo. El uso de fórmulas para automatizar cálculos repititivos ahorro tiempo y redujo la posibilidad de errores, lo cual es crucial en trabajos que utilizan un análisis numérico. 

Un aspecto también importante fue la creación de tablas dinámicas y gráficos. Estas herramientas me permitieron visualizar los datos de una forma más clara y comprensible, lo cuál facilita la interpretación de los resultados para poder identificar patrones. Por ejemplo, al obtener los datos ya graficados, pude observar de manera inmediata como los datos se comportaban entre ellos. 

A lo laego del proceso, me enfrenté a diversos desafíos los cuáles pusieron a prueba mis habilidades. Uno de esos desafios fue el manejo de grandes conjuntos de datos, para solucionar esto, aprendí a optimizar y utilizar funciones más eficientes que permiten procesar datos de manera más rápida. 

La experiencia adquirida en esta fase no solo tiene aplicaciones académicas, sino también prácticas en el mundo profesional. Las hojas de cálculo son una herramienta esencial en campos como la ingeniería, la economía, la administración y las ciencias, donde el manejo y análisis de datos son fundamentales. Aprender a utilizarlas de manera eficiente me ha dado una ventaja competitiva y me ha preparado para enfrentar problemas más complejos en el futuro.

En conclusión, esta fase de trabajo con hojas de cálculo de Excel ha sido una experiencia muy importante que me ha permitido desarrollar habilidades técnicas, mejorar mi capacidad de organización y precisión, y enfrentar desafíos que han fortalecido mi pensamiento crítico y analítico. Aunque al principio el manejo de estas herramientas puede parecer abrumador, con práctica y dedicación es posible dominarlas y aprovechar todo su potencial.

\newpage
\section{Catonga Tecla Daniel Isaí}


En esta primera fase, al resolver la ecuación para obtener los coeficientes de Fourier, me pareció interesante descubrir que existía una función específica para calcular estos coeficientes y, a partir de ellos, obtener la serie de Fourier correspondiente. Fue una experiencia buena para mí porque, en este caso, la ecuación a resolver no presentaba una gran dificultad, ya que se trataba de una función lineal. Esto facilitó considerablemente el proceso, permitiéndole comprender mejor la metodología sin enfrentar complicaciones excesivas.

Sin embargo, uno de los desafíos con los que me encontré fue el hecho de que, al no haber trabajado recientemente con cálculo diferencial e integral, en un principio olvidé algunos pasos clave, como la correcta aplicación de las reglas de integración y derivación. Esto hizo que tuviera que repasar varias fórmulas y recordar conceptos que no había utilizado en algún tiempo, lo que representó un reto adicional.
A pesar de ello, logré superar estas dificultades y resolver correctamente la ecuación dada, que en este caso correspondía a la función \(f(x)=6-2x \). 

Una de las cosas que aprendí de la transformada de Fourier es que permite expresar funciones mediante series trigonométricas. En particular, me llamó la atención su aplicación en la inteligencia artificial y el procesamiento de señales que es una materia que actualmente estoy cursando, por lo que me resultó útil conocer un poco más allá. En la inteligencia artificial, las series de Fourier son utilizadas para el análisis de datos, reducción de dimensionalidad y el reconocimiento de patrones en grandes volúmenes de información y esto es importante para mí ya que es un área en la que me quiero especializar . En el procesamiento de señales, juegan un papel crucial en la filtración y mejora de calidad de audio e imagen, así como en la transmisión eficiente de datos.

Además, este ejercicio me permitió reforzar mis habilidades matemáticas en integración y manipulación algebraica, lo que considero esencial para el correcto manejo de ecuaciones en aplicaciones más avanzadas. Me ayudó también a comprender la importancia de la precisión en los cálculos y la correcta interpretación de los resultados obtenidos.

En los cálculos realizados, es interesante observar cómo el número obtenido en cada iteración depende del valor de \(n\), que representa el número de la iteración. A medida que aumentamos el valor de \(n\), la aproximación de la función original se va refinando. Cada iteración nos proporciona un valor que se utiliza para calcular los armónicos correspondientes, los cuales son esenciales para reconstruir la función.

Este proceso iterativo nos permite generar una serie de armónicos que, al ser sumados, proporcionan una aproximación cada vez más precisa de la función que estamos tratando de resolver. Cuantos más términos se incluyan en la suma, mayor es la exactitud de la representación de la función original. De este modo, la serie de Fourier ofrece una herramienta poderosa para aproximar funciones complejas con una precisión controlable.

Para el manejo de Excel no tuve mayores dificultades para construir la tabla de valores y calcular cada una de las iteraciones. Gracias a mi experiencia previa trabajando con tablas, el proceso fue relativamente sencillo y no presentó complicaciones. Además, Excel facilitó la organización de los datos y la obtención de los resultados de manera rápida y clara.

Por otro lado, la visualización gráfica de los resultados fue una herramienta clave para entender mejor el comportamiento de la serie de Fourier. Ver cómo la aproximación mejora a medida que se agregan más iteraciones me permitió apreciar de manera más tangible cómo los términos de la serie se ajustan progresivamente a la función original. La representación gráfica no solo ayudó a confirmar la precisión de la aproximación, sino que también facilitó la interpretación de los resultados, haciendo que el análisis fuera mucho más intuitivo.

Después de realizar los cálculos, pude comprender mejor por qué es útil utilizar cómputo paralelo para resolver este tipo de problemas. Cuando se trata de funciones más complejas, el uso de cómputo paralelo se vuelve esencial para realizar las iteraciones necesarias de manera eficiente. Esto permite aproximar la función original con la mayor precisión posible, reduciendo el tiempo de cálculo y optimizando los recursos al trabajar con múltiples procesos simultáneamente.

Además, el cómputo paralelo no solo mejora la eficiencia, sino que también facilita el manejo de grandes cantidades de datos o funciones de alta complejidad. A medida que se aumentan los términos en la serie de Fourier o se analizan funciones de mayor dimensión, la capacidad de distribuir los cálculos entre varios núcleos de procesamiento se convierte en una ventaja significativa. Esto no solo acelera el proceso de aproximación, sino que también permite abordar problemas más grandes y complejos que de otro modo serían difíciles de manejar con un enfoque secuencial.

El desarrollo de la serie de Fourier para esta función representó un ejercicio valioso para mejorar mis conocimientos en análisis matemático y su aplicación en la ciencia y tecnología. Me brindó una nueva perspectiva sobre la importancia de las series de Fourier en el análisis de funciones y cómo estas herramientas pueden utilizarse en la resolución de problemas reales. En futuras ocasiones, me gustaría explorar cómo la transformada de Fourier puede aplicarse a proyectos prácticos dentro de la inteligencia artificial y el procesamiento de datos, ya que considero que su impacto en estas áreas es de gran relevancia.

En cuanto a la experiencia con la programación en lenguaje C para implementar los procesos paralelos, representó un desafío estimulante que me permitió aplicar mis conocimientos de programación en un contexto matemático avanzado. Al principio, la implementación de la paralelización mediante OpenMP requirió un período de adaptación, ya que debí familiarizarme con las directivas específicas para dividir eficientemente las tareas de cálculo entre múltiples hilos de ejecución. Particularmente, la correcta sincronización de los procesos para evitar condiciones de carrera y garantizar resultados precisos fue uno de los aspectos más desafiantes pero enriquecedores del ejercicio.

La traducción de las ecuaciones matemáticas de la serie de Fourier a código C me brindó una perspectiva valiosa sobre cómo las abstracciones matemáticas pueden materializarse en implementaciones computacionales eficientes. Observé cómo el rendimiento mejoraba significativamente al distribuir el cálculo de los diferentes coeficientes y términos de la serie entre varios núcleos del procesador, lo que reafirmó la importancia del paralelismo en aplicaciones de cálculo intensivo. Además, este ejercicio me permitió profundizar en conceptos de programación de bajo nivel, como la gestión de memoria y optimización de bucles, que son cruciales para el procesamiento eficiente de grandes volúmenes de datos.

Este componente práctico de programación complementó perfectamente el análisis teórico, permitiéndome visualizar de manera tangible cómo los conceptos matemáticos abstractos pueden transformarse en soluciones computacionales concretas. La experiencia adquirida con la programación en C para el cálculo de series de Fourier ha fortalecido mi interés en la intersección entre las matemáticas avanzadas y la computación de alto rendimiento, áreas que considero fundamentales para mi desarrollo profesional en el campo de la inteligencia artificial y el procesamiento de señales.

\newpage
\section{Padilla Sanchez Hariel}

En esta primera fase, el profesor a cargo de la materia nos asignó una función diferente a cada integrante del equipo. En mi caso, trabajé con la función lineal f(x)=6−4x. Al ser una función lineal, los cálculos se redujeron considerablemente en comparación con funciones de mayor grado, lo que facilitó la obtención de los coeficientes de Fourier. Sin embargo, aún encontré desafíos, ya que algunos conceptos de integración y álgebra que no había utilizado recientemente requirieron revisarlos para aplicarlos correctamente. Además, algunas expresiones trigonométricas eran nuevas para mí, y al no tomarlas en cuenta hubiera hecho que la resolución tomará más tiempo del necesario.

En cuanto al procedimiento de la serie de Fourier, aunque los conceptos ya los había estudiado en la materia de Análisis de Sistemas Digitales, volver a retomarlos fue algo que me gusto ya que hay veces que conceptos importantes no se les da continuidad para ver sus aplicaciones. Un aspecto interesante fue analizar la paridad de la función para determinar si podía simplificar los cálculos estableciendo ciertos coeficientes como cero. Sin embargo, en este caso, aunque algunos coeficientes resultaron ser cero, fue por la naturaleza de la función y no por su paridad (de la función f(x) = 6 - 4x en general).

Más allá de los cálculos, este ejercicio me permitió reflexionar sobre la importancia del análisis de Fourier en diversas disciplinas. En la ciencia, es fundamental para modelar fenómenos físicos, como la propagación de ondas y la conducción del calor. En la ingeniería, tiene aplicaciones en el procesamiento de señales, la acústica y el análisis estructural. En el ámbito tecnológico, su uso es indispensable en la compresión de datos y el procesamiento de imágenes, lo cual me resulta especialmente interesante, ya que es un área con la que interactuamos frecuentemente. Particularmente en la inteligencia artificial, las series de Fourier juegan un papel crucial en el reconocimiento de patrones y la optimización de algoritmos de aprendizaje.

Una de las aplicaciones que más me llamó la atención al investigar sobre la serie de Fourier fue su uso en inteligencia artificial, particularmente en el ámbito de la salud ya que estoy haciendo un proyecto sobre biotecnología. Su aplicación se ve justamente en las señales biomédicas en concreto me interesaron las señales electromiográficas (EMG). Las cuales sirven para detectar la actividad eléctrica de nuestro músculos, y contiene una gran cantidad de información de una persona analizando en el dominio de la frecuencia mediante transformadas de fourier, esto nos lleva a poder analizar y detectar patrones o mejorando modelos para la adaptación o personalización de prótesis u ortesis.

\newpage

Al llevar a cabo los cálculos en la hoja de cálculo, pude apreciar la importancia de la organización y sistematización de los datos, no solo para esta práctica, sino para la carrera de Inteligencia Artificial en general. En esta disciplina, el manejo adecuado de los datos es fundamental. Si bien Excel no es la herramienta más óptima para trabajos de gran escala, resulta útil para cálculos simples y de menor complejidad, como los de esta práctica. Al principio, no recordaba bien cómo fijar filas y columnas para replicar automáticamente los cálculos, lo que me llevó a invertir tiempo en reorganizar la tabla. Sin embargo, una vez solucionado, pude visualizar mejor cómo variaban los coeficientes según los términos de la serie y comprender de manera más intuitiva la aproximación de Fourier a la función original. Además, el uso de la hoja de cálculo facilitó la detección de errores, ya que los valores atípicos o inconsistentes resaltaban de inmediato al graficar los resultados.

Desde un punto de vista más amplio, esta experiencia me ayudó a comprender mejor la relevancia del análisis de Fourier en múltiples disciplinas. En ingeniería, es fundamental para el análisis de circuitos eléctricos y la transmisión de señales, ya que permite descomponer señales complejas en componentes más simples. En física, es una herramienta esencial para describir fenómenos ondulatorios, como el comportamiento de las ondas sonoras o la propagación de señales electromagnéticas. En mi caso particular, el aprendizaje de Fourier se vincula directamente con mis intereses en inteligencia artificial y biotecnología, ya que el análisis de señales es una parte crucial en el procesamiento de datos biomédicos, como las señales electromiográficas (EMG).

A nivel personal, esta fase del proyecto me permitió desarrollar mayor confianza en mis habilidades matemáticas y en mi capacidad para resolver problemas de manera estructurada. Aunque al principio algunos conceptos parecían complejos, aplicarlos en un caso concreto me ayudó a notar cómo se entrelazan distintas áreas del conocimiento, desde el cálculo hasta la programación. Además, el uso de herramientas computacionales para realizar los cálculos me permitió comprobar cómo la combinación de teoría y tecnología facilita el análisis de problemas complejos.

En conclusión, la experiencia de resolver la serie de Fourier para una función algebraica me permitió reforzar mis conocimientos matemáticos, mejorar mi manejo de herramientas computacionales y comprender la aplicabilidad de estos conceptos en problemas reales. Más allá del ejercicio académico, este análisis me dejó una reflexión importante: el poder de las matemáticas para modelar el mundo que nos rodea y su papel fundamental en el desarrollo de tecnologías avanzadas.

\newpage
El desarrollo de este proyecto permitió aplicar de manera práctica los conceptos fundamentales del cómputo paralelo utilizando procesos, memoria compartida y semáforos en un entorno basado en Linux. A través de la implementación de un programa en C que calcula los coeficientes de la serie de Fourier para la función f(x)=6−4xf(x)=6−4x, se logró dividir y distribuir el trabajo entre procesos hijos, lo cual favoreció la comprensión de la sincronización y la comunicación interprocesos mediante las herramientas que ofrece el sistema operativo, como shmget, shmat, semget, semop, entre otras.

Uno de los aprendizajes más relevantes fue entender la necesidad de controlar el acceso a la memoria compartida para evitar condiciones de carrera. En este proyecto se emplearon semáforos para asegurar que cada proceso hijo accediera de forma ordenada y segura a la región de memoria compartida, evitando así errores en el cálculo o la escritura concurrente de datos. La implementación y control de estos mecanismos representó un reto técnico importante, pero también una valiosa oportunidad para fortalecer habilidades en programación de sistemas y cómputo concurrente.

Además, este ejercicio permitió observar cómo una tarea computacional que puede parecer simple, como el cálculo de una serie matemática, se puede optimizar significativamente mediante la paralelización. La creación de procesos hijos que se encargan cada uno de una parte del trabajo refleja el potencial del procesamiento paralelo para mejorar el rendimiento, especialmente en aplicaciones de mayor escala o con cargas de trabajo intensivas.

Por otro lado, la experiencia también dejó en evidencia la importancia de una buena planificación en el diseño del algoritmo paralelo. Fue necesario definir con precisión la cantidad de términos a calcular, distribuirlos correctamente entre los procesos y sincronizar la escritura en la memoria compartida para obtener resultados coherentes y precisos. Este enfoque promueve una forma de pensar más estructurada y modular, aspectos esenciales en el desarrollo de software robusto y escalable.

En resumen, esta fase del proyecto no solo contribuyó al entendimiento de la teoría detrás del cómputo paralelo, sino que permitió vivenciarla a través de un caso práctico y funcional, afianzando los conocimientos y fortaleciendo competencias clave en programación concurrente, manejo de memoria compartida y control de sincronización. Esta experiencia representa un avance significativo en la formación académica y profesional en el área de sistemas operativos y programación paralela.


}